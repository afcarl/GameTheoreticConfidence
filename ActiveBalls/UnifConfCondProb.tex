\documentclass{article}
\usepackage[utf8]{inputenc}

\usepackage{amsthm}
\usepackage{amssymb}
\usepackage{amsmath}
\usepackage{color}

\usepackage{hyperref}
\usepackage{url}
\usepackage{times}
\usepackage[algo2e]{algorithm2e}

%\usepackage{fullpage}
%\usepackage{amsmath,amsfonts,amsthm,amssymb}
\usepackage{bbm}
\usepackage{graphics, graphicx, xcolor}
\usepackage{enumitem}
%\usepackage{verbatim}		% for misc commenting, etc.
\usepackage{stmaryrd}
\usepackage{float}
\usepackage[mathscr]{euscript}


\usepackage{geometry}
%% \geometry{a4paper,
%%   total={170mm,220mm},
%%   marginparwidth=80mm,
%% left=5mm,
%% right=85mm,
%% top=20mm,
%% }

% For algorithms
\usepackage{algorithm}
\usepackage{algorithmic}

\usepackage{hyperref}
\usepackage{cleveref}



\title{Uniform Convergence of Empirical Conditional Measures}
\author{Akshay Balsubramani, Yoav Freund, Shay Moran}

\newtheorem{theorem}{Theorem}[section]
\newtheorem{claim}{Claim}[section]
\newtheorem{corollary}{Corollary}[theorem]
\newtheorem{lemma}[theorem]{Lemma}
\newtheorem{assumption}[theorem]{assumption}
\newtheorem{definition}[theorem]{Definition}

\newtheorem{thm}{Theorem}%[section]
\newtheorem{lem}[thm]{Lemma}
\newtheorem{prop}[thm]{Proposition}
\newtheorem{cor}[thm]{Corollary}
\newtheorem{conj}[thm]{Conjecture}
\newtheorem{obs}[thm]{Observation}
\newtheorem{defn}[thm]{Definition}
\newtheorem{alg}{Algorithm}
\newtheorem{ass}{Assumption}
\newtheorem{examp}{Example}
\newtheorem{property}{Property}
\setcounter{MaxMatrixCols}{20}

\DeclareMathOperator{\id}{id}
\DeclareMathOperator{\tr}{tr}
\DeclareMathOperator*{\argmin}{arg\,min}
\DeclareMathOperator*{\argmax}{arg\,max}
\DeclareMathOperator{\sgn}{sgn}
\DeclareMathOperator{\Prtxt}{Pr}
\DeclareMathOperator{\var}{var}
\DeclareMathOperator{\poly}{poly}
\DeclareMathOperator{\polylog}{polylog}

\newcommand{\err}{\mbox{err}}
\newcommand{\X}{{\cal X}}
\newcommand{\Y}{{\cal Y}}
\newcommand{\D}{{\cal D}}
\newcommand{\B}{{\cal B}}
\newcommand{\x}{\vec{x}}
\newcommand{\y}{\vec{y}}
\newcommand{\vv}{\vec{v}}
\newcommand{\cc}{\vec{c}}

\newcommand{\K}{{\cal K}}
\newcommand{\restrictedto}{\triangleright}
\renewcommand{\SS}{{\cal S}} % Specialists
\newcommand{\CC}{{\cal C}}  % constraints

\newcommand{\outcome}{z}
\newcommand{\empoutcome}{\hat{\outcome}}
\newcommand{\polarity}{p}

\newcommand{\bd}[1]{\mathbf{#1}}  % for bolding symbols
\newcommand{\RR}{\mathbb{R}}      % Real numbers
\newcommand{\ZZ}{\mathbb{Z}}      % Integers
\newcommand{\NN}{\mathbb{N}}      % natural numbers
\newcommand{\RP}{\mathbb{RP}}      % real projective space
\newcommand{\Sp}{\mathbb{S}}
\newcommand{\HH}{\mathbb{H}}
\newcommand{\col}[1]{\left[\begin{matrix} #1 \end{matrix} \right]}
\newcommand{\comb}[2]{\binom{#1^2 + #2^2}{#1+#2}}
\newcommand{\vnorm}[1]{\left\lVert#1\right\rVert} % vector norm
\newcommand{\bfloor}[1]{\left\lfloor#1\right\rfloor} % floor function
\newcommand{\bceil}[1]{\left\lceil#1\right\rceil} % ceiling function
\newcommand{\ifn}{\mathbf{1}} % indicator function for sets
\newcommand{\EV}{\mathbb{E}} % expected value operator
\newcommand{\evp}[2]{\mathbb{E}_{#2} \left[#1\right]} % expected value operator
\newcommand{\abs}[1]{\left| #1 \right|}
\newcommand{\pr}[1]{\Prtxt \left(#1\right)}
\newcommand{\prp}[2]{\Prtxt_{#2} \left(#1\right)}
\newcommand{\ip}[2]{\left\langle #1, #2 \right\rangle}
\newcommand{\emperr}[2]{\widehat{\mbox{err}}_{#2} \left(#1\right)}
\newcommand{\Ex}{\mathbb{E}} % expected value operator


\newcommand{\pdis}[1]{P_{dis}\left(#1\right)}
\newcommand{\lrp}[1]{\left(#1\right)}
\newcommand{\lrb}[1]{\left[#1\right]}
\newcommand{\lrsetb}[1]{\left\{#1\right\}}

\newcommand{\corr}{\mbox{corr}}
\newcommand{\ones}[1]{\mathbbm{1}^{#1}}
\newcommand{\vA}{\mathbf{A}}
\newcommand{\va}{\mathbf{a}}
\newcommand{\vd}{\mathbf{d}} 
\newcommand{\vf}{\mathbf{f}}
\newcommand{\vF}{\mathbf{F}} 
\newcommand{\vI}{\mathbf{I}}  
\newcommand{\vh}{\mathbf{h}}
\newcommand{\vx}{\mathbf{x}}
\newcommand{\vb}{\mathbf{b}} 
\newcommand{\vu}{\mathbf{u}}   
\newcommand{\vl}{\mathbf{l}}
\newcommand{\vm}{\mathbf{m}}    
\newcommand{\vg}{\mathbf{g}}   
\newcommand{\vp}{\mathbf{p}}
\newcommand{\vq}{\mathbf{q}}
\newcommand{\vr}{\mathbf{r}}
\newcommand{\vs}{\mathbf{s}}
\newcommand{\vt}{\mathbf{t}}
\newcommand{\vw}{\mathbf{w}}
\newcommand{\vz}{\mathbf{z}}
\newcommand{\valpha}{\vec{\alpha}}
\newcommand{\vbeta}{\vec{\beta}}
\newcommand{\vzero}{\mathbf{0}}
\newcommand{\vone}{\mathbf{1}}

\newcommand{\cA}{\mathcal{A}}
\newcommand{\cB}{\mathcal{B}}
\newcommand{\cC}{\mathcal{C}}
\newcommand{\cH}{\mathcal{H}}
\newcommand{\cX}{\mathcal{X}}
\newcommand{\cY}{\mathcal{Y}}
\newcommand{\cZ}{\mathcal{Z}}
\newcommand{\cG}{\mathcal{G}}
\newcommand{\cD}{\mathcal{D}}
\newcommand{\cU}{\mathcal{U}}
\newcommand{\cS}{\mathcal{S}}
\newcommand{\cL}{\mathcal{L}}
\newcommand{\cN}{\mathcal{N}}
\newcommand{\cM}{\mathcal{M}}
\newcommand{\cF}{\mathcal{F}}
\newcommand{\cW}{\mathcal{W}}
\newcommand{\cE}{\mathcal{E}}
\newcommand{\cO}{\mathcal{O}}


\newcommand{\I}{\mathcal{I}}
\newcommand{\N}{\mathbb{N}}

\newcommand{\bias}{\text{bias}}
\newcommand{\ebias}{\widehat{\text{bias}}}

\newcommand{\eD}{\hat{\D}}
\newcommand{\ep}{\hat{p}}

\newcommand{\samp}{S}
\newcommand{\usamp}{\underline{S}}

\newcommand{\eps}{\epsilon}

\newcommand{\sign}{\text{sign}}
\newcommand{\new}[1]{\textcolor{red}{#1}}

\newcommand{\comment}[3]{\marginpar{\textcolor{#2}{#1: #3}}}
%\newcommand{\comment}[3]{}
\newcommand{\shay}[1]{\comment{Shay}{red}{#1}}
\newcommand{\yoav}[1]{\comment{Yoav}{blue}{#1}}
\newcommand{\akshay}[1]{\comment{Akshay}{magenta}{#1}}

\begin{document}

\maketitle

\section{Formal Statement}

Let $p$ be a distribution over $X$.
Let $\cB$ be a collection of events, and let $A$ be an event.
Assume that $\cB$ has a finite VC dimension, denoted by $d$.
Consider $n$ independent samples from~$p$, denoted by $x_1,\ldots,x_n$.
We would like to estimate $p(A \vert B)$ simultaneously for all $B\in \B$.
It is natural to consider the empirical estimates:
\[p_n(A\vert B)=\frac{\sum_i 1_{[x_i\in A \cap B]}}{\sum_i 1_{[x_i\in B]}}.\]
We study how well and when do these estimates approximate the underlying ones simultaneously.

To demonstrate the kind of statements we would like to derive,
consider the case where there is only one event $B$ in $\cB$, 
and let $k(B)=\sum_i 1_{[x_i\in B]}$.
A Chernoff bound implies that conditioned on the event that~$k(B)>0$, 
it holds with probability at least $1-\delta$ that:
\[\bigl\lvert p(A\vert B) - p_n(A \vert B) \bigr\rvert \leq \sqrt{\frac{2\ln(1/\delta)}{k(B)}}.\]
To derive it use that conditioned on $x_i\in B$ the event $x_i\in A$ has probability $p(A\vert B)$, 
and so conditioned on a specific value for $k(B)$ we get $k(B)$ i.i.d random coin tosses with bias $p(A\vert B)$.

Note that the bound we get on the probability is also a random variable, 
since it involves $k(B)$ in it, which is random.
We stress that this is the type of statement we want:
the more samples belong to $B$ | the more certain we are with the empirical estimate.

We would like to prove something like this:
\begin{theorem}[A too good to be true UCECM]\label{thm:toogood}
The following event occurs with probability at least $1-\delta$:
\[\bigl(\forall B\in\B\bigr):\bigl\lvert p(A \vert B) - p_n(A \vert B) \bigr\rvert \leq O\Bigl(\sqrt{\frac{d\ln(1/\delta)}{k(B)}}\Bigr),\]
where $k(B) = \sum_{i=1}^n 1[x_i\in B]$.
\end{theorem}
\Cref{thm:toogood} is, unfortunately, false. 
As an example, consider the probability space defined by drawing uniformly $j\sim[n]$,
and then coloring $j$ by $c_i\in\{\pm 1\}$ uniformly.
For each $i$ let $B_i$ denote the event that $i$ was chosen,
and let $A$ denote the event that the chosen color was  $+1$.
(formally, $B_i = \{i\}\times\{\pm 1\}$, and $A=[n]\times\{+1\}$).
One can verify that the VC dimension of $\B=\{B_i : i\leq n\}$ is $1$.

\Cref{thm:toogood} fails in this setting:
indeed, if we sample $n$ times from this space 
then with a constant probability there will be some  $j$
that is assigned with the same color in each sample,  say $+1$, 
and will be picked some $\Theta(\log n/\log\log n)$ times.
Therefore, $p_n(A\vert B_i) = 1$, $p(A\vert B_i)=1/2$,
and $1-(1/2)>> \sqrt{\log\log n/\log n}$.

We prove the following slightly weaker variant:
\begin{theorem}[UCECM]\label{thm:UCECM}
The following event occurs with probability at least $1-\delta$:
\[\bigl(\forall B\in\B\bigr):\bigl\lvert p(A \vert B) - p_n(A \vert B) \bigr\rvert \leq 
\sqrt{\frac{k_0}{k(B)}},\]
where $k_0 =200\bigl(d\log(2n) + \log(10/\delta)\bigr)$, and $k(B) = \sum_{i=1}^n 1[x_i\in B]$.
\end{theorem}

\paragraph{Discussion.}
\shay{Analyze this comparison more clearly.
I hope we can derive a statement like:
our bound becomes meaningful
for $B$ such that $p(B)\geq d\log n/n$
but the naive bound requires $p(B)\geq \sqrt{d/n}$.}

The uniform convergence bound in \Cref{thm:UCECM} is based on ``luckiness'':
if many points from the sample belong to $B$ then the confidence interval
around its (true) measure is smaller. 
In other words, the bound on the confidence interval is random.
It is worth noting that this can be combined with the classical uniform convergence
result due \cite{vapnik} to yield a bound on the confidence interval
that depends only on the true measure $p(B)$:
indeed, it implies that roughly $p(B)\pm \sqrt{d/n}$ 
empirical points from the sample are will indeed belong to $B$
(with high probability), and so, once $n$ is sufficiently large,
one can replace the radius of the confidence interval by 
$\Theta\Bigl(\sqrt{\frac{d\log n}{p(B)n - \sqrt{dn}}}\Bigr)$, 
which is for a large $n$ roughly $\sqrt{\frac{d\log n}{p(B)n}}$.

Another relevant remark is that a weaker statement than \Cref{thm:UCECM}
can be derived quite easily as a corollary of the the classical uniform convergence
result~\cite{vapnik}. 
Indeed, since the VC dimension of $\{B\cap A : i\in \I\}$ is at most $d$, it follows that 
\[p_n(A\vert B)\approx\frac{p(A\cap B) \pm d/\sqrt{n}}{p(B)\pm d/\sqrt{n}}.\]
Thus, a simple calculation yield that the radius of the confidence interval is at least 
$\Omega(\frac{d}{p(B)\sqrt{n}})$.
When comparing these bound, one sees that that our bound becomes non-trivial for $B$
once $p(B) \geq d\log n /n$, whereas the other one becomes non-trivial
only once $p(B)\geq d/\sqrt{n}$.


\section{Proof of \Cref{thm:UCECM}}

The idea is to follow the standard argument due to~\cite{vapnik} 
which derives the standard uniform convergence result (and to modify it accordingly). 
We follow the exposition of~\cite{anthony}:
consider a double-sample of size $2n$ from~$p$, denoted by $x_1,\ldots,x_{2n}$.
Let $S_1$ denote the first half of the sample and $S_2$ the second.
Define $E_1$ to be the event whose probability we want to bound:
\[E_1 = \Bigl\{ S_1\in X^n : (\exists B\in \B):~ 
\bigl\lvert p(A \vert B) - p_{n,1}(A \vert B) \bigr\rvert > 
\sqrt{\frac{k_0}{k_{1}(B)}} \Bigr\},\]
where $p_{n,1}$ is the empirical measure induced by $S_1$, 
and $k_{1}(B)=\sum_{i=1}^n 1[x_i\in B]$ ($p_{n,2}, k_{2}$ are defined similarly).
Let $E_2$ denote the event
\[E_2 = 
\Bigl\{
S_1S_2\in X^{2n} : (\exists B\in\B):~
\bigl\lvert p_{n,1}(A \vert B)   -  p_{n,2}(A \vert B) \bigr\rvert >  
\frac{1}{4}\sqrt{\frac{k_0}{k(B)}}
\Bigr\},
\]
where $k(B) = k_{1}(B)+k_{2}(B)$.
The strategy of showing that $\Pr[E_1]$ is by reducing it to showing that $\Pr[E_2]$ 
is small, and then to show that the latter is small using a standard \emph{symmetrization} argument. 
For the first part, we would like to argue like in~\cite{anthony}, that
\begin{equation}\label{eq:anthony} 
(\forall S_1\in E_1): \Pr_{S_2}[S_1S_2\in E_2]\geq \frac{1}{100},
\end{equation}
which would imply that $\Pr[E_1]\leq 100\Pr[E_2]$ and yield the reduction.
However, \Cref{eq:anthony} does not necessarily hold: indeed consider
$S_1\in E_1$ such that there is a single $B$ for which 
\[
\bigl\lvert p(A \vert B) - p_{n,1}(A \vert B) \bigr\rvert > 
\sqrt{\frac{k_0}{k_{1}(B)}}
\]
and further assume that 
(i) $B$ has a tiny measure, say $p(B) = 1/n$,
(ii) $p(A\vert B)= 1/2$, and
(iii) $p_{n,1}(A\vert B) = 0$.
Therefore, $k_2(B)=1$ with probability at least $1/4$
and therefore $p_{n,2}(A \vert B)=p_{n,1}(A \vert B)$
with probability at least $1/8$, which is the negation of $E_2$.

To get around this, we introduce an auxiliary event $F$, defined by
\[F = \Bigl\{S_1\in X^{n} : (\forall B\in\B): k_{1}(B) \geq k_0 \implies p(B)\geq \frac{k_{1}(B)}{4n}\Bigr\}. \]

Note that $F$ discards the problematic example from above 
(by constraining every $B$ that witnesses $E_1$ to have a large probability). 
As we will see, $F$ is a typical event that happens with high probability,
and so, it enables us to replace \Cref{eq:anthony} with the following
\begin{lemma}\label{lem:reduction}
If $\Pr[E_1]\geq \delta$ then 
\begin{enumerate}
\item $\Pr[E_1\cap F] \geq \frac{1}{2}\Pr[E_1]$, and
\item $(\forall S_1\in E_1\cap F): \Pr_{S_2}[S_1S_2\in E_2] \geq \frac{1}{8}$. 
\end{enumerate}
\end{lemma}
We defer the proof of \Cref{lem:reduction} to a later section, 
and assume it for now towards proving \Cref{thm:UCECM}.

\Cref{lem:reduction} yields the following win-win situation:
either $\Pr[E_1 \leq \delta]$ and we are done, 
or that $\Pr[E_1]\leq 16\Pr[E_2]$:
\[\frac{\Pr[E_2]}{\Pr[E_1]} \geq  \frac{1}{2}\frac{\Pr[E_2]}{\Pr[E_1\cap F]} \geq \frac{\Pr[E_2 \vert E_1\cap F]}{2} \geq \frac{1}{16},\]
where the first inequality uses the first item of \Cref{lem:reduction}, 
the second inequality follows by definition of conditional probability,
and the last inequality is implied by the the second item of \Cref{lem:reduction}.


We proceed to the standard symmetrization argument
that establishes $\Pr[E_2]\leq\delta/16$:
instead of sampling $S_1S_2\sim p^{2n}$,
consider the following equivalent process:
\begin{itemize}
\item[(i)] Sample $S\sim p^{2n}$.
\item[(ii)] Partition $S$ uniformly into two subsamples $S_1,S_2$, each of size $n$.
\end{itemize}
The following lemma implies that $\Pr[E_2]\leq \delta/16$, and finishes the proof.
\begin{lemma}\label{lem:e2}
For every $S\in X^{2n}$
\[\Pr_{S\to S_1S_2}\bigl[S_1S_2\in E_2\bigr]\leq \frac{\delta}{16},\]
where the randomness is over the uniform partitioning
of $S$ into $S_1,S_2$.
\end{lemma}
\qed

\subsection{Proof of \Cref{lem:reduction}}

\paragraph{Item {\it 1}.}
We begin with the first item;
it suffices to show that $\Pr[F]\geq 1-\delta/2$.
This follows from the standard uniform convergence bound with
the difference that of using the multiplicative Chernoff bound instead
of the additive bound:
Consider a double sample $S=(\samp_1,\samp_2)\sim p^{2n}$.
Let $E_1$ denote the event whose probability we want to bound:
\[
E_1 =\bigl\{ S_1\in X^n : (\exists {B\in\B}):
  k_{1}(B) \geq k_0 \mbox{ and } p(B) \leq \frac{k_{1}(B)}{4n}\bigr\}, 
\]
and let $E_2$ denote the event:
\[
E_2 = 
\bigl\{ S_1S_2\in X^{2n} : (\exists {B\in\B}):
  k_{1}(B) \geq k_0 \mbox{ and } k_{2}(B) \leq \frac{k_{1}(B)}{2}\bigr\}.
\]
The proof follows from the following two lemmas:
\begin{lemma}\label{lem:aux11}
$\Pr[E_1]\leq 10\Pr[E_2].$
\end{lemma}
\begin{lemma}\label{lem:aux12}
$\Pr[E_2]\leq \delta/20.$
\end{lemma}
\begin{proof}[Proof of Lemma~\ref{lem:aux11}]
It suffices to show that $\Pr[E_2 \vert E_1]\geq 1/10$.
Indeed, this would imply that 
$\Pr[E_1] \leq 10\Pr[E_1 \cap E_2]\leq 10\Pr[E_2]$.

Let $S_1\in E_1$. Since $S_2$ and $S_1$ are independent,
it suffices to show that 
\[\Pr_{S_2\sim p^n}\bigl[(S_1,S_2)\in E_2\bigr] \geq 1/10.\]
Let $B\in\B$ such that $p(B)\leq \frac{k_{1}(B)}{4n}$.
We want to show that $k_{2}(B)\leq \frac{k_{1}(B)}{2}$ with probability at least $1/10$.
We consider two cases:
(i) if $p(B) < 1/2n$
then we bound as follows:
\begin{align*}
\Pr\Bigl[ k_{2}(B) > \frac{k_{1}(B)}{2}\Bigr]
&\leq
\Pr\Bigl[ k_{2}(B) > 0 \Bigr]\\
&\leq np(B) < 1/2 < 9/10.
\end{align*}
(ii) Else, if $p(B) \geq 1/2n$, then by the multiplicative Chernoff bound (see e.g.\cite{} page ??):
\[
\Pr\Bigl[ k_{2}(B) > \frac{k_{1}(B)}{2}\Bigr]
\leq
\Pr\Bigl[ k_{2}(B) > 2p(B)\cdot n\Bigr]
\leq
\exp\Bigl( \frac{-p(B)\cdot n}{3}\Bigr)\leq \exp(-1/6)\leq 9/10.
\]
So, conditioned on $E_1$, 
the event $E_2$ occurs with probability at least $1/10$.

\begin{proof}[Proof of Lemma~\ref{lem:aux12}]

Instead of sampling $\samp_1$ and then $\samp_2$,
we first sample $\samp=\samp_1\cup \samp_2$ and 
then partition it to $\samp_1$ and $\samp_2$ uniformly.
Now, for a fixed $\samp$ what is the probability (over the random partition)
that $E_2$ occurs?
Let $\B|_{\samp} = \{B|_{\samp} : B\in\B\}$.
It suffices to show that the event
\[E_2|_{\samp} = 
\bigl\{ \{S_1,S_2\}\text{ is a partition of $S$ into two equal parts} :
\exists {B\in\B|_{\samp}}:
  k_{i,1}>k_0 \mbox{ and } k_{i,2} \leq \frac{k_{i,1}}{2}
  \bigr\}
\]
has probability at most $\delta/10$, for every every $\samp$ 
(where the probability is over the partition of $\samp$ into $\samp_1,\samp_2$).
To analyze this we use a union bound. 
We only need to consider $B$'s in $\B|_{\samp}$ such that $k(B) > k_0$,
where $k(B) = k_{1}(B)+ k_{2}(B)$.
Fix such a $B$;
without loss of generality assume that
first $k(B)$ points out of the~$2n$
points in $S$ are in $B$. 
For every $i\leq k(B)$,
let $X_i$ denote the indicator of the event
that the $i$'th point in $S$ was drawn into $S_2$.
Set~$X=\sum_{i=1}^{k(B)}X_i$.
Note that $B$ causes $E_2|_{\samp}$ to occur if and only if $X\leq k(B)/3$.
To analyze this event define a random variable $Y=\sum_{i=1}^{k(B)}Y_i$,
where the $Y_i$'s are independent Bernoulli random variables with probability $1/2$.
One can verify that $\Pr[X\geq k(B)/3]\leq \Pr[Y\geq k(B)/3]$,
and therefore it suffices to analyze the latter, 
which is a standard application of Chernoff bound:
\shay{Here we should refer to the statement that sampling without repetitions is more concentrated.}
\begin{align*}
\Pr\bigl[X\leq k(B)/3\bigr]&\leq 
\Pr\bigl[Y\leq k(B)/3\bigr]\\
&=\Pr\bigl[Y- k(B)/2 < -k(B)/6\bigr]\leq
\exp\bigl(-2(1/6)^2k(B)\bigr) < \exp(-k_0/100)
\end{align*}
(where the last inequality is because $k(B) > k_0$).
By Sauer's Lemma
the number of distinct restrictions $B|_S$ is at most $(2n)^d$, and therefore 
\[\Pr[E_2]\leq (2n^d)\exp(-k_0/100).\]
Having $k_0\geq 200\bigl(d\log(2n) + \log(10/\delta)\bigr)$ yields
that this probability is at most $\delta/20$.
\end{proof}

\paragraph{Item {\it 2}.}
We now derive the second item;
let $S_1\in E_1\cap F$; we want to show that $\Pr_{S_2}[S_1S_2\in E_2]\geq 1/8$.
Fix a $B$ such that $\bigl\lvert p_{n,1}(A\vert B) - p(A\vert B) \bigr\rvert > \sqrt{\frac{k_0}{k(B)}}$.
Thus, it follows that $k(B) > k_0$, and having $S_1\in F$ implies that $p(B)\geq \frac{k_1(B)}{4n} > \frac{k_0}{4n}$.
Therefore, by basic properties of the binomial distribution it follows that $k_2(B) \geq np(B)\geq \frac{k_1(B)}{4}$ 
with probability at least $1/4$, and that conditioned on this event (by Chernoff):
\[ \bigl\lvert p_{n,2}(A \vert B) - p(A\vert B) \bigr\rvert \leq \sqrt{\frac{2\ln(2)}{k_2(B)}} < \sqrt{\frac{50}{k(B)}} \]
with probability at least $1/2$. 
(where in the last inequality we used that if $k_{2}(B)\geq k_1(B)/4$ then $k(B) \leq 5k_2(B)$).
To summarize, 
with probability at least~$\frac{1}{8}=\frac{1}{2}\cdot\frac{1}{4}$ we have that
$\bigl\lvert p_{n,2}(A \vert B) - p(A\vert B) \bigr\rvert < \sqrt{\frac{50}{k(B)}}$,
and therefore
\[
\bigl\lvert p_{n,2}(A \vert B) - p_{n,1}(A\vert B) \bigr\rvert > 
\sqrt{\frac{k_0}{k(B)}} - \sqrt{\frac{50}{k(B)}}\geq 
\sqrt{\frac{k_0}{k(B)}} - \frac{1}{2}\sqrt{\frac{k_0}{k(B)}}=
\frac{1}{2}\sqrt{\frac{k_0}{k(B)}},
\]
with probability at least $1/8$, which which implies that $S_1S_2\in E_2$ with this probability.
\end{proof}


\subsection{Proof of \Cref{lem:e2}}

Let $S\in X^{2n}$. 
We need to show that
\[\Pr_{S\to S_1S_2}\Bigl[(\exists B\in \B|_S) : \bigl\lvert p_{n,1}(A \vert B) - p_{n,2}(A \vert B)  \bigr\rvert > \frac{1}{4}\sqrt{\frac{k_0}{k(B)}} \Bigr] \leq \frac{\delta}{16}.\]
To this end we show that
for every $B\in \B|_S$, 
\[\Pr_{S\to S_1S_2}\Bigl[\bigl\lvert p_{n,1}(A \vert B) - p_{n,2}(A \vert B)  \bigr\rvert > \frac{1}{4}\sqrt{\frac{k_0}{k(B)}} \Bigr] \leq \frac{\delta/16}{\lvert \B|_S\rvert},\]
which would finish the proof by a union bound.
Fix $B\in \B|_S$. 
We may assume that $k(B)> k_0$ (otherwise the above event has probability $0$). 
Denote by $k(A\cap B)$ the number of points in $S$
that are in $A\cap B$, and denote by $p_n(A\vert B) = \frac{k(A\cap B}{k(B)}$.
It suffices to show that
\[\Pr_{S\to S_1S_2}\Bigl[\bigl\lvert p_{n,1}(A \vert B) - p_{n}(A \vert B)  \bigr\rvert > \frac{1}{2}\cdot\frac{1}{4}\sqrt{\frac{k_0}{k(B)}} \Bigr] \leq \frac{1}{2}\cdot\frac{\delta/16}{\lvert \B|_S\rvert}.\]
Let $k_1(B)$ denote the number of points in $S_1$ that belong to $B$.
By \new{concentration},
\shay{Here we should refer to the statement that sampling without repetitions is more concentrated.}
\[ \Pr[k_1(B) < k(B)/2] \leq \exp(-k(B)/8) \leq \exp(-k_0/8),\]
and conditioned on that $k_1(B) \geq k(B)/2\geq k_0/2$, \new{by Chernoff}:
\[\Pr\Biggl[ \lvert p_{n,1}(A \vert B) - p_n(A \vert B) \rvert >\sqrt{\frac{2\log(1/\delta')}{k(B)/2}}~ \Biggr\vert~ k_1(B) \geq k(B)/2\Biggr] \leq \delta',
\]
for every $\delta'$. 
Thus, plugging $\delta'$ such that $\log(1/\delta') = k_0/256$ yields that
\begin{align*}
\Pr_{S\to S_1S_2}\Bigl[\bigl\lvert p_{n,1}(A \vert B) - p_{n}(A \vert B)  \bigr\rvert > \frac{1}{2}\cdot\frac{1}{4}\sqrt{\frac{k_0}{k(B)}} \Bigr] 
&\leq  \delta' + \exp(-k_0/8)\\
&\leq  \exp(-k_0/256) + \exp(-k_0/8)\\
&\leq 2\exp(-k_0/256).
\end{align*}
Now, $\lvert \B|_S\rvert \leq 2(2n)^d$ by Sauer's Lemma,  and therefore, by the definition of $k_0$,
this probability is at most $\frac{1}{2}\cdot\frac{\delta/16}{\lvert \B|_S\rvert}$, as required.

\qed

%\newpage
%
%\section{Ball Specialists}
%
%We restrict our attention to a special case which corresponds,
%roughly, to nearest neighbor methods.
%\begin{enumerate}
%\item The input space $\X$ is a finite set in $R^d$. We assume a
%  uniform distribution over $\X$.
%  \item We consider labels $y\in \{-1,+1\}$. For There is a fixed but
%    unknown conditional probability defined over $\X$, i.e. $P(Y=+1 |
%    X=\x)$. Our goal is to estimate whether $P(Y=+1|X=\x)$ is smaller
%    or larger than $1/2$.
%  \item
%    The notation is simpler if we use expected values instead of
%    probabilities. We use the term {\em bias} of a ball to refer to the conditional
%    expectation of the label for a ball by
%    \[
%    \bias(\x) \doteq P(Y=+1|X=\x) - P(Y=-1|X=\x)
%    \]
%  \item
%    The rules that we use are ``specialists'' that are balls. The set
%    $\B$ contains all rules of the form
%    \[
%    B_{r,\cc,s}(\x) =
%    \begin{cases}
%      s & \text{if } \| \cc- \x \| \leq r \\
%    0 & \text{otherwise }
%    \end{cases}
%    \]
%    Where $r \geq 0$ is the radius of the ball, $\cc \in R^d$ is the
%    center of the ball and $s \in \{-1,+1\}$ is the polarity of the ball.
%    We will drop the subscripts of $B$ when clear from context.
%  \item
%    We use $\x \in B$ to indicate that $B(\x) \neq 0$.
%  \item
%    We denote the {\em probability} of a ball $B$ by $p(B) \doteq
%    \frac{|B|}{|\X|}$.
%  \item
%    We define term {\em bias} of a ball to refer to the conditional
%    expectation of the label for a ball by
%    \[
%    \bias(B) \doteq E\left( y|\x \in B \right).
%    \]
%  \item We define a sample $\samp$ as a sequence of labeled examples:
%    \[\samp= \left\langle (\x_1,y_1),(\x_2,y_2),\ldots,(\x_m,y_m)
%    \right\rangle \]
%    Where $\x_i$ are chosen uniformly at random (with replacement)
%    from $\X$ and $y_i$ are chosen according to the (unknown)
%    conditional distribution of the label given $\x$.
%  \item Given a sample $\samp$, we define the number of instances in
%    $(\x,y) \in \samp$ such that $\x \in B$ to be the {\em size} of
%    $B$ according to the sample $\samp$ and denote it by $k_{\samp}$. We define
%    the {\em empirical probability} of the ball $B$ according to
%    $\samp$ by $p_{\samp}(B) \doteq k_{\samp}(B)/|\samp|$.
%  \item Given a sample $\samp$ we define the estimate of the
%    $\bias(B)$ to be
%    \[
%    \ebias_{\samp}(B) = \frac{\sum_{i=1}^m y_i B(\x_i)}{\sum_{i=1}^m B(\x_i)}
%    \]
%  \item Using uniform convergence bounds we define for each ball $B$
%    a confidence interval:
%    $[l,h]=[\ebias(B)-\Delta,\ebias(B)+\Delta]$.
%    if $l>0$ we say that $B$ imposes a {\em positive constraint}, if
%    $h<0$ we say that $B$ imposes a {\em negative constraint}, i
%    $l\leq 0 \leq h$ we say that $B$ does not impose a constraint.
%\end{enumerate}
%
%\section{Uniform convergence bounds for sepcialists.}
%
%\subsection{Uniform convergence of biases}
%
%We will use the following Lemma, which we prove in the appendix, in Section~\ref{sec:auxuc}.
%\begin{lemma}\label{lem:auxuc}
%Let $\B$ be a family of specialists of VC dimension $d$.
%Set $p_0 = \frac{100\bigl(d\log(2n) + \log(10/\delta)\bigr)}{n}$, 
%where $1/2\geq \delta>0$,
%and let $S=\bigl((x_i,y_i)\bigr) \sim p^n$.
%Then:
%\[
%\Pr
%\Bigl[
%\exists {B\in\B}: p_{\samp}(B) \geq p_0 \mbox{ and } p(B) \leq \frac{p_{\samp}(B)}{10}
%\Bigr]\leq \delta
%\]
%and
%\[
%\Pr
%\Bigl[
%\exists {B\in\B}: p(B) \geq p_0 \mbox{ and } p_{\samp}(B) \leq \frac{p(B)}{10}
%\Bigr]\leq \delta.
%\]
%\end{lemma}
%
%\begin{theorem} \label{thm:Bias-Convergence}
%Let $\B$ be a family of specialists of VC dimension $d$.
%Set $k_0 = 100\bigl(d\log(2n) + \log(10/\frac{\delta}{2})\bigr)$, where
% $1/2\geq \delta>0$,
%and let $S=\bigl((x_i,y_i)\bigr) \sim p^n$.
%Then:
%\[\Pr_{\samp\sim p^n}\left[\exists {B\in\B}:~\lvert
%  \bias_{\samp}(B) -  \bias(B)\rvert \geq \sqrt{\frac{100d\log n +
%      \log(1/\delta)}{k(B)} }
%  \mbox{ and } k(B)>k_0
%  \right] \leq \delta,
%%\binom{n}{\leq d}\exp\bigl(-2\eps^2n\bigr),
%\]
%where $k(B) = \lvert\{ i : x_i\in B \}\rvert$.
%%Then, the event:
%%?For every specialist S: the absolute difference between its empirical bias and its true bias of S is at most sqrt{ (10d log n + log (1/delta) )/ k }?
%%has probability at most delta.
%\end{theorem}
%
%\subsection{sampling without replacement}
%Before proving Theorem~\ref{thm:Bias-Convergence} we prove a 
%lemma regarding the concentration of sampling without replacement.
%
%Let $a_1,\ldots,a_n$ be a fixed sequence where $a_i \in
%\{0,1\}$. We denote the number of 1's by $k=\sum_{i=1}^n a_i$.
%
%We compare sampling from $a_1^n$ $j$ with and without replacement.
%Consider a sequence of $n$ draws made with or without replacement.
%We denote the number of 1's in the sequence after $i$ draws by
%$k_i$. This defines a sequence of random variables $k_1,k_2,\ldots,k_n$.
%
%We denote the fraction of 1's in $a_1^n$ by $p=k/n$.
%
%We denote by $P_{ij}$ the probability that $k_j=i$ when sampling {\em
%  with} replacement, and by $Q_{ji}$ the probability that $k_j=i$ when
%sampling {\em without} replacement.
%
%The mean of $k_i$ is the same whether the sampling is equal to $pi$
%whether sampling with replacement or not.
%
%\newcommand{\Pupper}{{\mathbf P}^{>}}
%\newcommand{\Plower}{{\mathbf P}^{<}}
%\newcommand{\Qupper}{{\mathbf Q}^{>}}
%\newcommand{\Qlower}{{\mathbf Q}^{<}}
%
%
%We are interested in the tail probabilities for the two processes. For
%$1 \leq i \leq n$ and $1 \leq j < \lfloor pi \rfloor $ we define the lower tails as:
%\[
%\Plower_{ij} = \sum_{l=1}^j P_{il},\;\;
%\Qlower_{ij} = \sum_{l=1}^j Q_{il}\;\;
%\]
%For $\lceil pi \rceil < j \leq i$ we define the upper tails to be
%\[
%\Pupper_{ij} = \sum_{l=j}^i P_{il},\;\;
%\Qupper_{ij} = \sum_{l=j}^i Q_{il}\;\;
%\]
%
%Using this notation we can state the following lemma. Intuitively,
%this lemma states that the distribution generated by sampling without
%replacement is more concentrated around the mean than the
%corresponding distribution for sampling with replacement.
%\begin{lemma}
%For all $1 \leq i \leq n$ and $1 \leq j < \lfloor pi \rfloor$,
%$\Plower_{ij} > \Qlower_{ij}$, and for all $1 \leq i \leq n$, $\lceil
%pi \rceil < j \leq i$, $\Pupper_{ij} > \Qupper_{ij}$.
%\end{lemma}
%\begin{proof}
%  By Induction over $i$.
%  \end{proof}
%
%\begin{proof} {\em of Theorem~\ref{thm:Bias-Convergence}} \newline
%We follow the double sampling argument.
%Consider a double sample $S=(\samp_1,\samp_2)\sim p^{2n}$.
%Let $E_1$ denote the event whose probability we want to bound:
%\[
%E_1 = \text{"}\exists {B\in\B}:~\lvert \bias_{\samp_1}(B) -
%\bias(B)\rvert \geq \sqrt{\frac{100d\log n + \log(1/\delta)}{k_1(B)}}
%\mbox{ and } k_1(B)>k_0
%\text{"}, 
%\]
%and let $E_2$ denote the event:
%\[
%E_2 = \text{"}\exists {B\in\B}:~\lvert \bias_{\samp_1}(B) -  \bias_{\samp_2}(B)\rvert \geq \sqrt{\frac{100d\log n + \log(1/\delta)}{k_1(B)}}
%  \mbox{ and }~ \bigl(k_1(B) + k_2(B)\bigr)>k_0
%  \text{"},
%\]
%where $k_i(b)$ is the number of examples in $S_i$ that belong to $B$.
%
%Now, assume towards contradiction that $\Pr[E_1] > \delta$.
%The following two lemmas yield a contradiction.
%\begin{lemma}\label{lem:aux1}
%If $\Pr[E_1] > \delta$ then $\Pr[E_1]\leq 10\Pr[E_2].$
%\end{lemma}
%\begin{lemma}\label{lem:aux2}
%$\Pr[E_2]\leq \delta/10.$
%\end{lemma}
%
%\begin{proof}[Proof of Lemma~\ref{lem:aux1}.]
%
%\new{It suffices to show that $\Pr[E_2 \vert E_1]\geq \new{1/4}$.
%Indeed, this would imply that $\Pr[E_2 \vert E_1]\geq \new{1/4}$, 
%namely that $\Pr[E_1] \leq \new{4}\Pr[E_1 \land E_2]\leq 4\Pr[E_2]$.
%We do it in two steps:
%Consider the event
%\[
%F = \text{"}\forall {B\in\B}:~k_1(B) \geq k_0 \implies k_2(B) \geq k_1(B)/\new{20}
%  \text{"}.
%\]
%We first prove that $\Pr[F \vert E_1] \geq \new{1/2}$ and then that $\Pr[E_2 \vert E_1\cap F]\geq 1/2$.
%This concludes the proof as $\Pr[E_2 \vert E_1] = \Pr[F \vert E_1]\cdot\Pr[E_2 \vert E_1\cap F]$.}
%\new{
%We begin with the first step.
%Indeed, Lemma~\ref{lem:auxuc} and the choice of $k_0$ 
%imply that $\Pr[\lnot F] < \delta/2$.
%Therefore, $\Pr[\lnot F \vert E_1] \leq \Pr[\lnot F]/\Pr[E_1] \leq (\delta/2)/\delta\leq 1/2$
%(note that we used here the assumption that $\Pr[E_1]\geq \delta$).}
%
%To be continued...
%
%\end{proof}
%
%\begin{proof}[Proof of Lemma~\ref{lem:aux2}]\ \\
%\paragraph{High level.}
%Instead of sampling $S_1$ and then $S_2$,
%we first sample $S=S_1\cup S_2$ and 
%then partition it to $S_1$ and $S_2$ uniformly.
%Now, given a sample $S$ what is the probability (over the random partition)
%that $E_2$ occurs?
%First, we only need to worry about $B$'s such that $k_S(B) > 2k_0$.
%For such a $B$, the probability that the biases differ should be sufficiently
%small such that when we union bound over all $O(n^d)$ different $B$'s
%we still get that the resulting probability is at most $\delta$.
%
%\end{proof}
%
%%We distinguish between two cases: $p(B)$ is large and $p(B)$ is small.
%%The case when $p(B)$ is sufficiently large (at least $1/n$) is fine (I think).
%%
%%On the other hand, if $p(B)<k_0/n$ then, with hight probability
%%$k_1(B)<2k_0$ and the ball will be eliminated from consideration.
%
%\iffalse
%Further assume that $\bias(B) = 0.8$,  that $\bias_{\samp_1}(B)=0$, and that $k_1(B)$
%is sufficiently large so the $\lvert \bias_{\samp_1}(B) -  \bias(B)\rvert \geq \sqrt{\frac{100d\log n + \log(1/\delta)}{k_1(B)} }$.
%Now, since $p(B) \leq 1/1000n$ it follows that $k_2(B)=0$ with probability at least $0.99$.
%Now, it is plausible to define $\bias_{\samp_2}(B)=0$ when $k_2(B)=0$,
%and therefore we get that $\bias_{\samp_2}=\bias_{\samp_1}=0$ with probability at least $0.99$ and so the event $E_2$ occurs with probability less than $0.01$,
%unlike what we wanted.
%\fi
%\end{proof}
%
%\section{The algoithm}
%
%\subsection{Defining safe sets}
%
%Given a set of constraints, and given a polarity  $s \in
%\{-1,+1\}$ we define a point $\x$ as a $s$-safe if
%the following holds
%\begin{itemize}
%\item There is an $s$ constraint that contains $\x$.
%\item For any $-s$ constraint $C$ that contains $\x$, there exists an
%  $s$ constraint $D$ such that $D \subset C$. 
%\end{itemize}
%
%Points that are neither positive nor negative safe are called
%``unsafe''
%
%\subsection{Active Learning}
%
%At each stage of Active Learning we query the label of two sets, each
%of size $n$
%\begin{itemize}
%\item {\bf Uniform Set :} select points uniformly at random from the whole
%  domain.
%\item {\bf Active Set :} select points uniformaly at random from the unsafe
%  set.
%\end{itemize}
%
%The Uniform sets from different stages are combined to form one
%large label set. The active set from each stage is considered
%separately.
%
%The bias of each constraint is calculated with respect to the
%cumulative uniform sample, and with respect to each one of the active
%sets. Each of these sets defines a contraint on a subset of the $z_i$'s 
%
%\section{Sufficient conditions on the true bias}
%\newcommand{\Bayes}{f^*}
%The goal of our algorithm is to generate a rule $f$ that
%approximates the Bayes decision rule $\Bayes(\x) = \sign(\bias(\x))$.
%Specifically, we set our goal to be minimizing the regret $\err(f) - \err(\Bayes)$.
%
%In this section we define conditions on $\bias(x)$ that guarantee that
%the algorithm will achieve a regret of $\epsilon$ after querying the
%labels of $O(\log 1/\epsilon)$ instances.
%
%We do this in two steps. First, we describe a technical condition on
%$\bias(\x)$. Second, we give two more natural conditions that
%guarantee the technical condition.
%
%\subsection{$\epsilon$ approximation using Balls}
%
%Let $\D$ be the distribution over $\X\times \{-1,+1\}$ where the
%distribution over $\X$ is uniform and the conditional distribution
%over the label $y$ given $\x$ is define by $\bias(\x)$.
%
%\newcommand{\Be}{{\cal B}_{\epsilon}}
%\newcommand{\fBe}{f_{\Be}}
%\newcommand{\UBe}{U_{\epsilon}}
%We say that $\D$ is $\epsilon$ approximable using balls for $\epsilon>0$ if
%there exists a collection of balls $\Be$ such that 
%\begin{itemize}
%\item For all $B \in \Be$, $P_{\D}(B)>\epsilon$
%\item For all $B \in \Be$, $|\bias(B)-1/2| > \epsilon$
%\item For all $B \in \Be$ and for any point $\x \in B$, $\sign(\bias(\x))=\sign(\bias(B))$
%\item %Let $\UBe$ be the union of all of the balls in $\Be$, i.e. $\UBe
%  %\doteq \bigcup_{B \in \Be} B$\\
%  Define the prediction function $\fBe$ as follows:
%  \[
%  \fBe(\x) \doteq \begin{cases}
%    1 & \exists B \in \Be, \;\;\mbox{such that}\;\; \x \in B,
%    \bias(B)>0 \\
%    0 & \nexists B \in \Be \;\;\mbox{such that}\;\; \x \in B \\
%    -1 & \exists B \in \Be, \;\;\mbox{such that}\;\; \x \in B,
%    \bias(B)<0
%  \end{cases}
%  \]
%  The final requirement is that $\err(\fBe) \leq \err(\Bayes)+\epsilon$
%\end{itemize}
%
%\subsection{Specific conditions}
%
%I believe either of the following conditions implies
%$\epsilon$-approximation using balls.
%
%\begin{enumerate}
%\item
%  {\bf The conditional probability is Lipshitz smooth}. In other words, for
%any $\x,\y \in \X$:
%\[
%|\bias(\x) - \bias(\y)| \leq \alpha \|\x-\y\|_2^{\beta}
%\]
%\item
%The bias is bounded away from zero and {\bf the boundary is low
%dimensional}. More technically:
%\begin{itemize}
%\item There exists some $\epsilon_0$ such that $\forall \x \in \X$, $|\bias(\x)|>\epsilon_0$.
%\item Recall that $\X \subset R^d$. Define boundary balls as balls
%  that contain and element with positive bias and an element with
%  negative bias. Let $G_{\epsilon}$ be the union of all boundary balls
%  with probability at most $\epsilon$. We say that the boundary has
%  low dimension if $P_{\D}(G_{\epsilon}) \to 0$ as $\epsilon \to 0$
%  (one has to be a bit careful in the definition here because $\X$ is finite, so
%  $G_{1/2|\X|}=\emptyset$ trivially).  
%\end{itemize}
%\end{enumerate}
%
%
%\appendix
%
%%\section{Proof of Lemma~\ref{lem:auxuc}}\label{sec:auxuc}
%%\begin{proof}
%%We use the standard uniform convergence bound with
%%the difference that we use the multiplicative Chernoff bound instead
%%of the additive bound. 
%%We follow the double sampling argument.
%%Consider a double sample $S=(\samp_1,\samp_2)\sim p^{2n}$.
%%Let $E_1$ denote the event whose probability we want to bound:
%%\[
%%E_1 = \text{"}\exists {B\in\B}:
%%  k_1(B) \geq t_0 \mbox{ and } p(B) \leq \frac{k_1(B)}{4n}
%%\text{"}, 
%%\]
%%and let $E_2$ denote the event:
%%\[
%%E_2 = \text{"}\exists {B\in\B}:
%%  k_1(B) \geq t_0 \mbox{ and }k_2(B) \leq \frac{k_1(B)}{2}
%%\text{"}.
%%\]
%%The proof follows from the following two lemmas:
%%\begin{lemma}\label{lem:auxuc1}
%%\[\Pr[E_1]\leq 10\Pr[E_2].\]
%%\end{lemma}
%%\begin{lemma}\label{lem:auxuc2}
%%\[\Pr[E_2]\leq \delta/10.\]
%%\end{lemma}
%%\begin{proof}[Proof of Lemma~\ref{lem:auxuc1}]
%%It suffices to show that $\Pr[E_2 \vert E_1]\geq 1/10$.
%%Indeed, this would imply that 
%%$\Pr[E_1] \leq 10\Pr[E_1 \land E_2]\leq 10\Pr[E_2]$.
%%
%%Let $S_1\in E_1$. Since $S_2$ and $S_1$ are independent,
%%it suffices to show that 
%%\[\Pr_{S_2\sim p^n}\bigl[(S_1,S_2)\in E_2\bigr] \geq 1/10.\]
%%Let $B\in\B$ such that $p(B)\leq \frac{k_1(B)}{4n}$.
%%We want to show that $k_2(B)\leq \frac{k_1(B)}{2}$ with probability at least $1/10$.
%%We consider two cases:
%%(i) if $p(B) < 1/2n$
%%then we bound as follows:
%%\begin{align*}
%%\Pr\Bigl[ k_2(B) > \frac{k_1(B)}{n}\Bigr]
%%&\leq
%%\Pr\Bigl[ k_2(B) > 0 \Bigr]\\
%%&\leq np(B) < 1/2 < 9/10.
%%\end{align*}
%%(ii) Else, if $p(B) \geq 1/2n$, then by the multiplicative Chernoff bound (see e.g.\cite{} page ??):
%%\[
%%\Pr\Bigl[ k_2(B) > \frac{k_1(B)}{2}\Bigr]
%%\leq
%%\Pr\Bigl[ p_{\samp_2}(B) > 2p(B)\Bigr]
%%\leq
%%\exp\Bigl( \frac{-p(B)\cdot n}{3}\Bigr)\leq \exp(-1/6)\leq 9/10.
%%\]
%%So, conditioned on $E_1$, 
%%the event $E_2$ occurs with probability at least $1/10$.
%%
%%
%%\end{proof}
%%
%%\begin{proof}[Proof of Lemma~\ref{lem:auxuc2}]
%%
%%Instead of sampling $\samp_1$ and then $\samp_2$,
%%we first sample $\samp=\samp_1\cup \samp_2$ and 
%%then partition it to $\samp_1$ and $\samp_2$ uniformly.
%%Now, for a fixed $\samp$ what is the probability (over the random partition)
%%that $E_2$ occurs?
%%Let $\B|_{\samp} = \{B|_{\samp} : B\in\B\}$.
%%It suffices to show that the event
%%\[E_2|_{\samp} = \text{"}\exists {B\in\B|_{\samp}}:
%%  k_1(B)>t_0 \mbox{ and } k_2(B) \leq \frac{k_1(B)}{2}
%%\text{"}\]
%%has probability at most $\delta/10$, for every every $\samp$ 
%%(where the probability is over the partition of $\samp$ into $\samp_1,\samp_2$).
%%To analyze this we use a union bound. 
%%We only need to consider $B$'s in $\B|_{\samp}$ such that $k(B) > t_0$.
%%Fix such a $B$.
%%Without loss of generality assume that
%%first $k(B)$ points out of the $2n$
%%points in $S$ are in $B$. 
%%For every $i\leq k(B)$,
%%let $X_i$ denote the indicator of the event
%%that the $i$'th point in $S$ was drawn into $S_2$.
%%Set~$X=\sum_{i=1}^{k(B)}X_i$.
%%Note that $B$ causes $E_2|_{\samp}$ to occur if and only if $X\leq k(B)/3$.
%%To analyze this event define a random variable $Y=\sum_{i=1}^{k(B)}Y_i$,
%%where the $Y_i$'s are independent Bernoulli random variables with probability $1/2$.
%%One can verify that $\Pr[X\geq k(B)/3]\leq \Pr[Y\geq k(B)/3]$,
%%and therefore it suffices to analyze the latter, 
%%which is a standard application of Chernoff bound:
%%\begin{align*}
%%\Pr\bigl[X\leq k(B)/3\bigr]&\leq 
%%\Pr\bigl[Y\leq k(B)/3\bigr]\\
%%&=\Pr\bigl[Y- k(B)/2 < -k(B)/6\bigr]\leq
%%\exp\bigl(-2(1/6)^2k(B)\bigr) < \exp(-k/100)
%%\end{align*}
%%(where the last inequality is because $k(B) > t_0$).
%%By Sauer's Lemma
%%the number of distinct restrictions $B|_S$ is at most $(2n)^d$, and therefore 
%%\[\Pr[E_2]\leq (2n^d)\exp(-k/100).\]
%%Setting $k\geq 100\bigl(d\log(2n) + \log(10/\delta)\bigr)$ yields
%%that this probability is at most $\delta/10$.
%%\end{proof}
%%\end{proof}
%%
%
%
%
\end{document}
