\documentclass{article}
\usepackage[utf8]{inputenc}

\usepackage{amsthm}
\usepackage{amssymb}
\usepackage{amsmath}
\usepackage{color}

\usepackage{hyperref}
\usepackage{url}
\usepackage{times}
\usepackage[algo2e]{algorithm2e}

%\usepackage{fullpage}
%\usepackage{amsmath,amsfonts,amsthm,amssymb}
\usepackage{bbm}
\usepackage{graphics, graphicx, xcolor}
\usepackage{enumitem}
%\usepackage{verbatim}		% for misc commenting, etc.
\usepackage{stmaryrd}
\usepackage{float}
\usepackage[mathscr]{euscript}


\usepackage{geometry}
%% \geometry{a4paper,
%%   total={170mm,220mm},
%%   marginparwidth=80mm,
%% left=5mm,
%% right=85mm,
%% top=20mm,
%% }

% For algorithms
\usepackage{algorithm}
\usepackage{algorithmic}


\title{Non-Parametric Active Learning}
\author{Akshay Balsubramani, Yoav Freund, Shay Moran}

\newtheorem{theorem}{Theorem}[section]
\newtheorem{claim}{Claim}[section]
\newtheorem{corollary}{Corollary}[theorem]
\newtheorem{lemma}[theorem]{Lemma}
\newtheorem{assumption}[theorem]{assumption}
\newtheorem{definition}[theorem]{Definition}

\newtheorem{thm}{Theorem}%[section]
\newtheorem{lem}[thm]{Lemma}
\newtheorem{prop}[thm]{Proposition}
\newtheorem{cor}[thm]{Corollary}
\newtheorem{conj}[thm]{Conjecture}
\newtheorem{obs}[thm]{Observation}
\newtheorem{defn}[thm]{Definition}
\newtheorem{alg}{Algorithm}
\newtheorem{ass}{Assumption}
\newtheorem{examp}{Example}
\newtheorem{property}{Property}
\setcounter{MaxMatrixCols}{20}

\DeclareMathOperator{\id}{id}
\DeclareMathOperator{\tr}{tr}
\DeclareMathOperator*{\argmin}{arg\,min}
\DeclareMathOperator*{\argmax}{arg\,max}
\DeclareMathOperator{\sgn}{sgn}
\DeclareMathOperator{\Prtxt}{Pr}
\DeclareMathOperator{\var}{var}
\DeclareMathOperator{\poly}{poly}
\DeclareMathOperator{\polylog}{polylog}

\newcommand{\err}{\mbox{err}}
\newcommand{\X}{{\cal X}}
\newcommand{\Y}{{\cal Y}}
\newcommand{\D}{{\cal D}}
\newcommand{\B}{{\cal B}}
\newcommand{\x}{\vec{x}}
\newcommand{\y}{\vec{y}}
\newcommand{\vv}{\vec{v}}
\newcommand{\cc}{\vec{c}}

\newcommand{\K}{{\cal K}}
\newcommand{\restrictedto}{\triangleright}
\renewcommand{\SS}{{\cal S}} % Specialists
\newcommand{\CC}{{\cal C}}  % constraints

\newcommand{\outcome}{z}
\newcommand{\empoutcome}{\hat{\outcome}}
\newcommand{\polarity}{p}

\newcommand{\bd}[1]{\mathbf{#1}}  % for bolding symbols
\newcommand{\RR}{\mathbb{R}}      % Real numbers
\newcommand{\ZZ}{\mathbb{Z}}      % Integers
\newcommand{\NN}{\mathbb{N}}      % natural numbers
\newcommand{\RP}{\mathbb{RP}}      % real projective space
\newcommand{\Sp}{\mathbb{S}}
\newcommand{\HH}{\mathbb{H}}
\newcommand{\col}[1]{\left[\begin{matrix} #1 \end{matrix} \right]}
\newcommand{\comb}[2]{\binom{#1^2 + #2^2}{#1+#2}}
\newcommand{\vnorm}[1]{\left\lVert#1\right\rVert} % vector norm
\newcommand{\bfloor}[1]{\left\lfloor#1\right\rfloor} % floor function
\newcommand{\bceil}[1]{\left\lceil#1\right\rceil} % ceiling function
\newcommand{\ifn}{\mathbf{1}} % indicator function for sets
\newcommand{\EV}{\mathbb{E}} % expected value operator
\newcommand{\evp}[2]{\mathbb{E}_{#2} \left[#1\right]} % expected value operator
\newcommand{\abs}[1]{\left| #1 \right|}
\newcommand{\pr}[1]{\Prtxt \left(#1\right)}
\newcommand{\prp}[2]{\Prtxt_{#2} \left(#1\right)}
\newcommand{\ip}[2]{\left\langle #1, #2 \right\rangle}
\newcommand{\emperr}[2]{\widehat{\mbox{err}}_{#2} \left(#1\right)}

\newcommand{\pdis}[1]{P_{dis}\left(#1\right)}
\newcommand{\lrp}[1]{\left(#1\right)}
\newcommand{\lrb}[1]{\left[#1\right]}
\newcommand{\lrsetb}[1]{\left\{#1\right\}}

\newcommand{\corr}{\mbox{corr}}
\newcommand{\ones}[1]{\mathbbm{1}^{#1}}
\newcommand{\vA}{\mathbf{A}}
\newcommand{\va}{\mathbf{a}}
\newcommand{\vd}{\mathbf{d}} 
\newcommand{\vf}{\mathbf{f}}
\newcommand{\vF}{\mathbf{F}} 
\newcommand{\vI}{\mathbf{I}}  
\newcommand{\vh}{\mathbf{h}}
\newcommand{\vx}{\mathbf{x}}
\newcommand{\vb}{\mathbf{b}} 
\newcommand{\vu}{\mathbf{u}}   
\newcommand{\vl}{\mathbf{l}}
\newcommand{\vm}{\mathbf{m}}    
\newcommand{\vg}{\mathbf{g}}   
\newcommand{\vp}{\mathbf{p}}
\newcommand{\vq}{\mathbf{q}}
\newcommand{\vr}{\mathbf{r}}
\newcommand{\vs}{\mathbf{s}}
\newcommand{\vt}{\mathbf{t}}
\newcommand{\vw}{\mathbf{w}}
\newcommand{\vz}{\mathbf{z}}
\newcommand{\valpha}{\vec{\alpha}}
\newcommand{\vbeta}{\vec{\beta}}
\newcommand{\vzero}{\mathbf{0}}
\newcommand{\vone}{\mathbf{1}}

\newcommand{\cA}{\mathcal{A}}
\newcommand{\cB}{\mathcal{B}}
\newcommand{\cC}{\mathcal{C}}
\newcommand{\cH}{\mathcal{H}}
\newcommand{\cX}{\mathcal{X}}
\newcommand{\cY}{\mathcal{Y}}
\newcommand{\cZ}{\mathcal{Z}}
\newcommand{\cG}{\mathcal{G}}
\newcommand{\cD}{\mathcal{D}}
\newcommand{\cU}{\mathcal{U}}
\newcommand{\cS}{\mathcal{S}}
\newcommand{\cL}{\mathcal{L}}
\newcommand{\cN}{\mathcal{N}}
\newcommand{\cM}{\mathcal{M}}
\newcommand{\cF}{\mathcal{F}}
\newcommand{\cW}{\mathcal{W}}
\newcommand{\cE}{\mathcal{E}}
\newcommand{\cO}{\mathcal{O}}

\newcommand{\bias}{\text{bias}}
\newcommand{\ebias}{\widehat{\text{bias}}}

\newcommand{\eD}{\hat{\D}}
\newcommand{\ep}{\hat{p}}

\newcommand{\samp}{S}
\newcommand{\usamp}{\underline{S}}

\newcommand{\eps}{\epsilon}

\newcommand{\sign}{\text{sign}}
\newcommand{\new}[1]{\textcolor{red}{#1}}

\newcommand{\comment}[3]{\marginpar{\textcolor{#2}{#1: #3}}}
%\newcommand{\comment}[3]{}
\newcommand{\shay}[1]{\comment{Shay}{red}{#1}}
\newcommand{\yoav}[1]{\comment{Yoav}{blue}{#1}}
\newcommand{\akshay}[1]{\comment{Akshay}{magenta}{#1}}

\begin{document}

\maketitle
\section{Introduction}

The analysis of Active Learning using Muffler seems to run into a dead
end. Specifically, the elegant local condition we call ``Pairwise
safe'' is insufficient to guarantee version space consistency.

In this paper we suggest a different approach to the problem of active
learning using balls. Instead of using the min/max solution given by
Muffler, we consider an Occam's razor approach by which we make the
simplest prediction possible given the constraints.

\section{Ball Specialists}

We restrict our attention to a special case which corresponds,
roughly, to nearest neighbor methods.
\begin{enumerate}
\item The input space $\X$ is a finite set in $R^d$. We assume a
  uniform distribution over $\X$.
  \item We consider labels $y\in \{-1,+1\}$. For There is a fixed but
    unknown conditional probability defined over $\X$, i.e. $P(Y=+1 |
    X=\x)$. Our goal is to estimate whether $P(Y=+1|X=\x)$ is smaller
    or larger than $1/2$.
  \item
    The notation is simpler if we use expected values instead of
    probabilities. We use the term {\em bias} of a ball to refer to the conditional
    expectation of the label for a ball by
    \[
    \bias(\x) \doteq P(Y=+1|X=\x) - P(Y=-1|X=\x)
    \]
  \item
    The rules that we use are ``specialists'' that are balls. The set
    $\B$ contains all rules of the form
    \[
    B_{r,\cc,s}(\x) =
    \begin{cases}
      s & \text{if } \| \cc- \x \| \leq r \\
    0 & \text{otherwise }
    \end{cases}
    \]
    Where $r \geq 0$ is the radius of the ball, $\cc \in R^d$ is the
    center of the ball and $s \in \{-1,+1\}$ is the polarity of the ball.
    We will drop the subscripts of $B$ when clear from context.
  \item
    We use $\x \in B$ to indicate that $B(\x) \neq 0$.
  \item
    We denote the {\em probability} of a ball $B$ by $p(B) \doteq
    \frac{|B|}{|\X|}$.
  \item
    We define term {\em bias} of a ball to refer to the conditional
    expectation of the label for a ball by
    \[
    \bias(B) \doteq E\left( y|\x \in B \right).
    \]
  \item We define a sample $\samp$ as a sequence of labeled examples:
    \[\samp= \left\langle (\x_1,y_1),(\x_2,y_2),\ldots,(\x_m,y_m)
    \right\rangle \]
    Where $\x_i$ are chosen uniformly at random (with replacement)
    from $\X$ and $y_i$ are chosen according to the (unknown)
    conditional distribution of the label given $\x$.
  \item Given a sample $\samp$, we define the number of instances in
    $(\x,y) \in \samp$ such that $\x \in B$ to be the {\em size} of
    $B$ according to the sample $\samp$ and denote it by $k_{\samp}$. We define
    the {\em empirical probability} of the ball $B$ according to
    $\samp$ by $p_{\samp}(B) \doteq k_{\samp}(B)/|\samp|$.
  \item Given a sample $\samp$ we define the estimate of the
    $\bias(B)$ to be
    \[
    \ebias_{\samp}(B) = \frac{\sum_{i=1}^m y_i B(\x_i)}{\sum_{i=1}^m B(\x_i)}
    \]
  \item Using uniform convergence bounds we define for each ball $B$
    a confidence interval:
    $[l,h]=[\ebias(B)-\Delta,\ebias(B)+\Delta]$.
    if $l>0$ we say that $B$ imposes a {\em positive constraint}, if
    $h<0$ we say that $B$ imposes a {\em negative constraint}, i
    $l\leq 0 \leq h$ we say that $B$ does not impose a constraint.
\end{enumerate}

\section{Uniform convergence bounds for sepcialists.}
\subsection{Uniform convergence of measures}

\yoav{I think a multiplicative bound on the measure of the ghost
  sample would be more useful for bounding the deviation of the bias.}

\begin{theorem}\label{thm:uc1}
Let $\B$ be a family of specialists of VC dimension $d$,
let $\eps,\delta>0$, 
and let $S_1,S_2$ be two independent samples $\bigl((x_i,y_i)\bigr) \sim p^n$. Then:
\[\Pr_{\samp_1,\samp_2 \sim p^{2n}}\Bigl[\exists {B\in\B}:
  k_{\samp_2}(B) < \frac{k_{\samp_1}(B)}{8}  \Bigr] \leq 2n^d \log(k_{\samp_1}(B)) e^{-\frac{1}{3} k_{\samp_1}(B)}
\]
\end{theorem}

\begin{proof}
  Sketch of proof. We use Chernoff bound instead of Hoeffding bound to
  bound the ratio.

  We assume the true value of $p(B)$ to be of the form $2^{-i}$ for
  some $i$ and prove that as the ratio is 8, there exists some $i$ for which
  $p_{\samp_1}(B) \geq 2p(B)$ and $p_{\samp_2}(B) \leq p(B)/2$
\end{proof}

\iffalse
\begin{theorem}\label{thm:uc1}
Let $\B$ be a family of specialists of VC dimension $d$,
let $\eps,\delta>0$, 
and let $S=\bigl((x_i,y_i)\bigr) \sim p^n$. Then:
\[\Pr_{\samp\sim p^n}\Bigl[\exists {B\in\B}:~\lvert p_{\samp}(B) -  p(B)\rvert \geq \sqrt{\frac{100d + \log(1/\delta)}{n}} \Bigr] \leq \delta.
%\binom{n}{\leq d}\exp\bigl(-2\eps^2n\bigr),
\]
\end{theorem}
\fi

\subsection{Uniform convergence of biases}
\begin{theorem}
Let $\B$ be a family of specialists of VC dimension $d$, 
let $\eps,\delta>0$,
and let $S=\bigl((x_i,y_i)\bigr) \sim p^n$.
Then:
\[\Pr_{\samp\sim p^n}\Bigl[\exists {B\in\B}:~\lvert \bias_{\samp}(B) -  \bias(B)\rvert \geq \sqrt{\frac{100d\log n + \log(1/\delta)}{k(B)} }\Bigr] \leq \delta,
%\binom{n}{\leq d}\exp\bigl(-2\eps^2n\bigr),
\]
where $k(B) = \lvert\{ i : x_i\in B \}\rvert$.
%Then, the event:
%?For every specialist S: the absolute difference between its empirical bias and its true bias of S is at most sqrt{ (10d log n + log (1/delta) )/ k }?
%has probability at most delta.
\end{theorem}

\begin{proof}
We follow the double sampling argument.
Consider a double sample $S=(\samp_1,\samp_2)\sim p^{2n}$.
Let $E_1$ denote the event whose probability we want to bound:
\[
E_1 = \text{"}\exists {B\in\B}:~\lvert \bias_{\samp_1}(B) -  \bias(B)\rvert \geq \sqrt{\frac{100d\log n + \log(1/\delta)}{k_1(B)} }\text{"}, 
\]
,and let $E_2$ denote the event:
\[
E_2 = \text{"}\exists {B\in\B}:~\lvert \bias_{\samp_1}(B) -  \bias_{\samp_2}(B)\rvert \geq \sqrt{\frac{100d\log n + \log(1/\delta)}{\min(k_1(B),k_2(B))} }\text{"},
\]
where $k_i(b)$ is the number of examples in $S_i$ that belong to $B$.
The proof follows from the following two lemmas:
\begin{lemma}\label{lem:aux1}
\[\Pr[E_1]\leq 4\Pr[E_2]\]
\end{lemma}
\begin{lemma}
\[\Pr[E_2]\leq \delta\]
\end{lemma}

\paragraph{Proof of Lemma~\ref{lem:aux1}.}
%First note that we may restrict ourselves to the event
%that $k_1(B)$, $k_2(b)$, and $n\cdot p(b)$ are all within distance
%$\sqrt{n(100d+\log(1/\delta))}$ from each other, simultanously for every $B\in\B$.
%Indeed, by Theorem~\ref{thm:uc1} the probability it is not the case is at most $2\delta$.
%In the remaining of this proof we condition on this event 
%but for ease of notation omit spelling it out explicitly.

%\yoav{I plan to follow the steps in Devroye, Gyorfi, Lugosi Proof of
%  the Borel Cantelli theorem}

It suffices to show that $\Pr[E_2 \vert E_1]\geq 1/4$.
Indeed, this would imply that $\Pr[E_2 \vert E_1]\geq 1/4$, 
namely that $\Pr[E_1] \leq 4\Pr[E_1 \land E_2]\leq 4\Pr[E_2]$.

Let $S_1\in E_1$. Since $S_2$ and $S_1$ are independent,
it suffices to show that 
\[\Pr_{S_2\sim p^n}\bigl[(S_1,S_2)\in E_2\bigr] \geq 1/4.\]
Let $B\in\B$ such that $\lvert \bias_{\samp_1}(B) -  \bias(B)\rvert \geq \sqrt{\frac{100d\log n + \log(1/\delta)}{k_1(B)} }$.
We distinguish between two cases: $p(B)$ is large and $p(B)$ is small.
The case when $p(B)$ is sufficiently large (at least $1/n$) is fine (I think).
Therefore, assume $p(B)$ is small, say $p(B) \leq 1/1000n$.
Further assume that $\bias(B) = 0.8$,  that $\bias_{\samp_1}(B)=0$, and that $k_1(B)$
is sufficiently large so the $\lvert \bias_{\samp_1}(B) -  \bias(B)\rvert \geq \sqrt{\frac{100d\log n + \log(1/\delta)}{k_1(B)} }$.
Now, since $p(B) \leq 1/1000n$ it follows that $k_2(B)=0$ with probability at least $0.99$.
Now, it is plausible to define $\bias_{\samp_2}(B)=0$ when $k_2(B)=0$,
and therefore we get that $\bias_{\samp_2}=\bias_{\samp_1}=0$ with probability at least $0.99$ and so the event $E_2$ occurs with probability less than $0.01$,
unlike what we wanted.
\end{proof}

\section{The algoithm}

\subsection{Defining safe sets}

Given a set of constraints, and given a polarity  $s \in
\{-1,+1\}$ we define a point $\x$ as a $s$-safe if
the following holds
\begin{itemize}
\item There is an $s$ constraint that contains $\x$.
\item For any $-s$ constraint $C$ that contains $\x$, there exists an
  $s$ constraint $D$ such that $D \subset C$. 
\end{itemize}

Points that are neither positive nor negative safe are called
``unsafe''

\subsection{Active Learning}

At each stage of Active Learning we query the label of two sets, each
of size $n$
\begin{itemize}
\item {\bf Uniform Set :} select points uniformly at random from the whole
  domain.
\item {\bf Active Set :} select points uniformaly at random from the unsafe
  set.
\end{itemize}

The Uniform sets from different stages are combined to form one
large label set. The active set from each stage is considered
separately.

The bias of each constraint is calculated with respect to the
cumulative uniform sample, and with respect to each one of the active
sets. Each of these sets defines a contraint on a subset of the $z_i$'s 






\end{document}
